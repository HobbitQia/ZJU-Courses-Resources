% Format: If-Then structure
\usepackage{ifthen}

% Format: Code beautify

\usepackage{xcolor}
\usepackage{listings}
\definecolor{vgreen}{RGB}{104,180,104}
\definecolor{vblue}{RGB}{49,49,255}
\definecolor{vorange}{RGB}{255,143,102}

\lstdefinestyle{verilog-style}
{
    language=Verilog,
    basicstyle=\small\ttfamily,
    keywordstyle=\color{vblue},
    identifierstyle=\color{black},
    commentstyle=\color{vgreen},
    numbers=left,
    numberstyle=\tiny\color{black},
    numbersep=10pt,
    tabsize=8,
    breaklines,
    moredelim=*[s][\colorIndex]{[}{]},
    literate=*{:}{:}1
}

\makeatletter
\newcommand*\@lbracket{[}
\newcommand*\@rbracket{]}
\newcommand*\@colon{:}
\newcommand*\colorIndex{%
    \edef\@temp{\the\lst@token}%
    \ifx\@temp\@lbracket \color{black}%
    \else\ifx\@temp\@rbracket \color{black}%
    \else\ifx\@temp\@colon \color{black}%
    \else \color{vorange}%
    \fi\fi\fi
}
\makeatother
% Old Format: code highlight
% \lstset{xleftmargin=1em,xrightmargin=1em}
% \lstset{
%     framexleftmargin=0.5em,
%     framexrightmargin=1em,
%     framextopmargin=0em,
%     basicstyle=\footnotesize,
%     framexbottommargin=1em
% }
% \lstset{
%     backgroundcolor=\color{grey},
%     commentstyle=\color{darkgreen},
%     keywordstyle=\color{blue},
%     %     caption=\lstname,
%     basicstyle=\ttfamily\footnotesize,
%     breaklines=true,
%     columns=flexible,
%     mathescape=fause,
% %    backgroundcolor=\color{backcolour},   
% %    commentstyle=\color{codegreen},
% %    keywordstyle=\color{magenta},
% %    numberstyle=\tiny\color{codegray},
% %    stringstyle=\color{codepurple},
% %    basicstyle=\footnotesize,
% %    breakatwhitespace=false,         
% %    breaklines=true,                 
% %    captionpos=b,                    
% %    keepspaces=true,                 
% %    numbers=left,                    
% %    numbersep=5pt,                  
% %    showspaces=false,                
% %    showstringspaces=false,
% %    showtabs=false,                  
%     tabsize=4,
%     numbers=left,
%     stepnumber=1,
%     numberstyle=\small,
%     numbersep=1em
% }
% \lstloadlanguages{
%     C,
%     C++,
%     Java,
%     Matlab,
%     Python,
%     Bash,
%     Mathematica
% }
% \renewcommand{\lstlistingname}{代码}
% \usepackage{caption}
% \DeclareCaptionFont{white}{\color{white}}
% \DeclareCaptionFormat{listing}{\colorbox[cmyk]{0.43, 0.35, 0.35,0.01}{\parbox{\textwidth}{\hspace{15pt}#1#2#3}}}
% \captionsetup[lstlisting]{format=listing,labelfont=white,textfont=white, singlelinecheck=false, margin=0pt, font={bf,footnotesize}}
% \usepackage{listings}
% \lstset{
%     basicstyle=\footnotesize\ttfamily, % Standardschrift
%     %numbers=left,               % Ort der Zeilennummern
%     numberstyle=\tiny,          % Stil der Zeilennummern
%     %stepnumber=2,               % Abstand zwischen den Zeilennummern
%     numbersep=5pt,              % Abstand der Nummern zum Text
%     tabsize=4,                  % Groesse von Tabs
%     extendedchars=true,         %
%     breaklines=true,            % Zeilen werden Umgebrochen
%     keywordstyle=\color{red},
%     frame=b,         
%     escapeinside=``,
%     keywordstyle=[1]\textbf,    % Stil der Keywords
%     keywordstyle=[2]\textbf,    %
%     keywordstyle=[3]\textbf,    %
%     keywordstyle=[4]\textbf,    %\sqrt{\sqrt{}} 
%     %stringstyle=\color{white}\ttfamily, % Farbe der String
%     showspaces=false,           % Leerzeichen anzeigen ?
%     showtabs=false,             % Tabs anzeigen ?
%     xleftmargin=17pt,
%     framexleftmargin=17pt,
%     framexrightmargin=5pt,
%     framexbottommargin=4pt,
%     %backgroundcolor=\color{lightgray},
%     commentstyle=\color{green}, % comment color
%     keywordstyle=\color{blue},  % keyword color
%     stringstyle=\color{red},    % string color
%     showstringspaces=false      % Leerzeichen in Strings anzeigen ?        
% }
% \lstloadlanguages{% Check Dokumentation for further languages ...
%     %[Visual]Basic
%     %Pascal
%     C,
%     C++,
%     Python,
%     Java
% }

\usepackage{underscore}